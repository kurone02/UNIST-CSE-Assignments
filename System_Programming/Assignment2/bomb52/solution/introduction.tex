
\addcontentsline{toc}{section}{Introduction}
\section*{Introduction}
To begin with, let's first look at the source code \textbf{bomb.c}
\begin{minted}[frame=single,framesep=10pt]{c}
  input = read_line();             /* Get input                   */
  phase_1(input);                  /* Run the phase               */
  phase_defused();                 /* Drat!  They figured it out!
            * Let me know how they did it. */
  printf("Phase 1 defused. How about the next one?\n");

  /* The second phase is harder.  No one will ever figure out
    * how to defuse this... */
  input = read_line();
  phase_2(input);
  phase_defused();
  printf("That's number 2.  Keep going!\n");

  /* I guess this is too easy so far.  Some more complex code will
    * confuse people. */
  input = read_line();
  phase_3(input);
  phase_defused();
  printf("Halfway there!\n");

  /* Oh yeah?  Well, how good is your math?  Try on this saucy problem! */
  input = read_line();
  phase_4(input);
  phase_defused();
  printf("So you got that one.  Try this one.\n");
  
  /* Round and 'round in memory we go, where we stop, the bomb blows! */
  input = read_line();
  phase_5(input);
  phase_defused();
  printf("Good work!  On to the next...\n");

  /* This phase will never be used, since no one will get past the
    * earlier ones.  But just in case, make this one extra hard. */
  input = read_line();
  phase_6(input);
  phase_defused();
\end{minted}

We can see that the bomb has six phases, and each phase requires us to input a certain string to defuse. There are 6 functions correspoding to 6 phases: \begin{verbatim} phase_1, phase_2, phase_3, phase_4, phase_5, phase_6 \end{verbatim}

\newpage
